% Generated by Sphinx.
\documentclass[letterpaper,10pt,english]{manual}
\usepackage[utf8]{inputenc}
\usepackage[T1]{fontenc}
\usepackage{babel}
\usepackage{times}
\usepackage[Bjarne]{fncychap}
\usepackage{longtable}
\usepackage{sphinx}


\title{rhipe Documentation}
\date{September 11, 2009}
\release{0.5}
\author{Saptarshi Guha}
\newcommand{\sphinxlogo}{}
\renewcommand{\releasename}{Release}
\makeindex
\makemodindex
\newcommand\PYGZat{@}
\newcommand\PYGZlb{[}
\newcommand\PYGZrb{]}
\newcommand\PYGaz[1]{\textcolor[rgb]{0.00,0.63,0.00}{#1}}
\newcommand\PYGax[1]{\textcolor[rgb]{0.84,0.33,0.22}{\textbf{#1}}}
\newcommand\PYGay[1]{\textcolor[rgb]{0.00,0.44,0.13}{\textbf{#1}}}
\newcommand\PYGar[1]{\textcolor[rgb]{0.73,0.38,0.84}{#1}}
\newcommand\PYGas[1]{\textcolor[rgb]{0.25,0.44,0.63}{\textit{#1}}}
\newcommand\PYGap[1]{\textcolor[rgb]{0.00,0.44,0.13}{\textbf{#1}}}
\newcommand\PYGaq[1]{\textcolor[rgb]{0.38,0.68,0.84}{#1}}
\newcommand\PYGav[1]{\textcolor[rgb]{0.00,0.44,0.13}{\textbf{#1}}}
\newcommand\PYGaw[1]{\textcolor[rgb]{0.13,0.50,0.31}{#1}}
\newcommand\PYGat[1]{\textcolor[rgb]{0.73,0.38,0.84}{#1}}
\newcommand\PYGau[1]{\textcolor[rgb]{0.32,0.47,0.09}{#1}}
\newcommand\PYGaj[1]{\textcolor[rgb]{0.00,0.44,0.13}{#1}}
\newcommand\PYGak[1]{\textcolor[rgb]{0.14,0.33,0.53}{#1}}
\newcommand\PYGah[1]{\textcolor[rgb]{0.00,0.13,0.44}{\textbf{#1}}}
\newcommand\PYGai[1]{\textcolor[rgb]{0.73,0.38,0.84}{#1}}
\newcommand\PYGan[1]{\textcolor[rgb]{0.13,0.50,0.31}{#1}}
\newcommand\PYGao[1]{\textcolor[rgb]{0.25,0.44,0.63}{\textbf{#1}}}
\newcommand\PYGal[1]{\textcolor[rgb]{0.00,0.44,0.13}{\textbf{#1}}}
\newcommand\PYGam[1]{\textbf{#1}}
\newcommand\PYGab[1]{\textit{#1}}
\newcommand\PYGac[1]{\textcolor[rgb]{0.25,0.44,0.63}{#1}}
\newcommand\PYGaa[1]{\textcolor[rgb]{0.19,0.19,0.19}{#1}}
\newcommand\PYGaf[1]{\textcolor[rgb]{0.25,0.50,0.56}{\textit{#1}}}
\newcommand\PYGag[1]{\textcolor[rgb]{0.13,0.50,0.31}{#1}}
\newcommand\PYGad[1]{\textcolor[rgb]{0.00,0.25,0.82}{#1}}
\newcommand\PYGae[1]{\textcolor[rgb]{0.13,0.50,0.31}{#1}}
\newcommand\PYGaZ[1]{\textcolor[rgb]{0.25,0.44,0.63}{#1}}
\newcommand\PYGbf[1]{\textcolor[rgb]{0.00,0.44,0.13}{#1}}
\newcommand\PYGaX[1]{\textcolor[rgb]{0.25,0.44,0.63}{#1}}
\newcommand\PYGaY[1]{\textcolor[rgb]{0.00,0.44,0.13}{#1}}
\newcommand\PYGbc[1]{\textcolor[rgb]{0.78,0.36,0.04}{#1}}
\newcommand\PYGbb[1]{\textcolor[rgb]{0.00,0.00,0.50}{\textbf{#1}}}
\newcommand\PYGba[1]{\textcolor[rgb]{0.02,0.16,0.45}{\textbf{#1}}}
\newcommand\PYGaR[1]{\textcolor[rgb]{0.25,0.44,0.63}{#1}}
\newcommand\PYGaS[1]{\textcolor[rgb]{0.13,0.50,0.31}{#1}}
\newcommand\PYGaP[1]{\textcolor[rgb]{0.05,0.52,0.71}{\textbf{#1}}}
\newcommand\PYGaQ[1]{\textcolor[rgb]{0.78,0.36,0.04}{\textbf{#1}}}
\newcommand\PYGaV[1]{\textcolor[rgb]{0.25,0.50,0.56}{\textit{#1}}}
\newcommand\PYGaW[1]{\textcolor[rgb]{0.05,0.52,0.71}{\textbf{#1}}}
\newcommand\PYGaT[1]{\textcolor[rgb]{0.73,0.38,0.84}{#1}}
\newcommand\PYGaU[1]{\textcolor[rgb]{0.13,0.50,0.31}{#1}}
\newcommand\PYGaJ[1]{\textcolor[rgb]{0.56,0.13,0.00}{#1}}
\newcommand\PYGaK[1]{\textcolor[rgb]{0.25,0.44,0.63}{#1}}
\newcommand\PYGaH[1]{\textcolor[rgb]{0.50,0.00,0.50}{\textbf{#1}}}
\newcommand\PYGaI[1]{\fcolorbox[rgb]{1.00,0.00,0.00}{1,1,1}{#1}}
\newcommand\PYGaN[1]{\textcolor[rgb]{0.73,0.73,0.73}{#1}}
\newcommand\PYGaO[1]{\textcolor[rgb]{0.00,0.44,0.13}{#1}}
\newcommand\PYGaL[1]{\textcolor[rgb]{0.02,0.16,0.49}{#1}}
\newcommand\PYGaM[1]{\colorbox[rgb]{1.00,0.94,0.94}{\textcolor[rgb]{0.25,0.50,0.56}{#1}}}
\newcommand\PYGaB[1]{\textcolor[rgb]{0.25,0.44,0.63}{#1}}
\newcommand\PYGaC[1]{\textcolor[rgb]{0.33,0.33,0.33}{\textbf{#1}}}
\newcommand\PYGaA[1]{\textcolor[rgb]{0.00,0.44,0.13}{#1}}
\newcommand\PYGaF[1]{\textcolor[rgb]{0.63,0.00,0.00}{#1}}
\newcommand\PYGaG[1]{\textcolor[rgb]{1.00,0.00,0.00}{#1}}
\newcommand\PYGaD[1]{\textcolor[rgb]{0.00,0.44,0.13}{\textbf{#1}}}
\newcommand\PYGaE[1]{\textcolor[rgb]{0.25,0.50,0.56}{\textit{#1}}}
\newcommand\PYGbg[1]{\textcolor[rgb]{0.44,0.63,0.82}{\textit{#1}}}
\newcommand\PYGbe[1]{\textcolor[rgb]{0.40,0.40,0.40}{#1}}
\newcommand\PYGbd[1]{\textcolor[rgb]{0.25,0.44,0.63}{#1}}
\newcommand\PYGbh[1]{\textcolor[rgb]{0.00,0.44,0.13}{\textbf{#1}}}
\begin{document}

\maketitle
\tableofcontents



Mainpage

\resetcurrentobjects
\hypertarget{--doc-installation}{}

\chapter{Setting up RHIPE}


\section{Requirements}
\begin{enumerate}
\item {} 
\emph{Protobuffers}

RHIPE uses Google's Protobuf library for serialization. This(the C/C++
libraries) must be installed on \emph{all} machines (master/workers). Get
Protobuffers from \href{http://code.google.com/p/protobuf/}{http://code.google.com/p/protobuf/}. RHIPE already has the
protobuf jar file inside it.
\begin{description}
\item[Non Standard Locations]
If installing protobuf to a non standard location, update the
PKG\_CONFIG\_PATH variable, e.g

\begin{Verbatim}[commandchars=@\[\]]
export PKG@_CONFIG@_PATH@PYGbe[=]@$PKG@_CONFIG@_PATH:@$CUSTROOT@PYGbe[/]lib@PYGbe[/]pkgconfig@PYGbe[/]
\end{Verbatim}

\end{description}

\item {} 
\emph{R} , tested on 2.8

\item {} 
\emph{rJava} The R package needs rJava.

\end{enumerate}

Tested on RHEL Linux, though \emph{may} work on Windows


\section{Installation}

On every machine

\begin{Verbatim}[commandchars=@\[\]]
R CMD INSTALL Rhipe@_VERSION.tar.gz
\end{Verbatim}

To load it

\begin{Verbatim}[commandchars=@\[\]]
library(Rhipe)
\end{Verbatim}

\resetcurrentobjects
\hypertarget{--doc-rhlapply}{}

\chapter{The \texttt{rhlapply} Command}


\section{Introduction}

\code{rhapply} applies a user defined function to the elements of a given
R list or the function can be run over the set of numbers from 1 to
n. In the former case the list is written to a sequence file,whose length is the
default setting of \code{rhwrite}.

Running a hundreds of thousadands of seperate trials
can be terribly inefficient, instead consider grouping them, i.e set
\code{mapred.max.tasks} to a value much smaller than the length of the
list.


\section{Return Value}

\code{rhlapply} returns a list, the names of which is equal to the names
of the input list (if given).


\section{Function Usage}

\begin{Verbatim}[commandchars=@\[\],numbers=left,firstnumber=1,stepnumber=1]
rhlapply @PYGbe[@textless[]-] @PYGap[function]( ll@PYGbe[=]@PYGaD[NULL],
                     fun,
                     ifolder@PYGbe[=]@PYGaB["]@PYGaB["],
                     ofolder@PYGbe[=]@PYGaB["]@PYGaB["],
                     readIn@PYGbe[=]T,
                     inout@PYGbe[=]c('lapply','sequence')
                     mapred@PYGbe[=]list()
                     setup@PYGbe[=]@PYGaD[NULL],jobname@PYGbe[=]@PYGaB["]@PYGaB[rhlapply"],...
                     )
\end{Verbatim}

Description follows
\begin{description}
\item[\code{ll}]
The list object, optional. Applies \code{fun} to \code{ll{[}{[}i{]}{]}} .
If instead \code{ll} is a numeric, applies \code{fun} to each element of
\code{seq(1,ll)}. If not given, must provide a value for \code{ifolder}

\item[\code{fun}]
A function that takes only one argument.

\item[\code{ifolder}]
If \code{ll} is null, provide a source here. Also change the value of
\code{inout{[}1{]}} to either \code{text} or \code{sequence}.

\item[\code{readIn}]
The results are stored in a temporary sequence file on the DFS which is
deleted. Should the results be returned in a list? Default is TRUE. For
large number of output key-values (e.g 1MM) set this to FALSE, using the
default options to \code{rhread} is extremely slow.

\item[\code{ofolder}]
If given the results are written to this folder and not deleted. If not,
they are written to temporary folder, read back in (assuming \code{readIn}
is TRUE) and deleted.

\item[\code{mapred}]
Options passed onto \code{rhmr}

\item[\code{setup}]
And expression that is called before running \code{func}. Called once per
JVM.

\item[\code{...}]
passed onto RHMR.

\end{description}


\subsection{RETURN}

An object that is passed onto \code{rhex}.

\resetcurrentobjects
\hypertarget{--doc-rhmr}{}

\chapter{The \texttt{rhmr} Command}


\section{Introduction}

The \code{rhmr} command runs a general mapreduce program using user supplied map
and reduce commands.


\section{Return Value}

In general a set of files on the Hadoop Distributed File System. It can be of
Text Format or a Sequence file format. In case of the latter, the key and values
can be any R data structure.


\section{Function}

\begin{Verbatim}[commandchars=@\[\],numbers=left,firstnumber=1,stepnumber=1]
rhmr @PYGbe[@textless[]-] @PYGap[function](map,reduce@PYGbe[=]@PYGaD[NULL],
         combiner@PYGbe[=]F, @PYGaf[@#CANNOT BE CHANGED]
         setup@PYGbe[=]@PYGaD[NULL],
         cleanup@PYGbe[=]@PYGaD[NULL],
         ofolder@PYGbe[=]'',
         ifolder@PYGbe[=]'',
         inout@PYGbe[=]c(@PYGaB["]@PYGaB[text"],@PYGaB["]@PYGaB[text"]),
         mapred@PYGbe[=]@PYGaD[NULL],
         shared@PYGbe[=]c(),
         jarfiles@PYGbe[=]c(),
         copyFiles@PYGbe[=]F,
         opts@PYGbe[=]rhoptions(),jobname@PYGbe[=]@PYGaB["]@PYGaB["])
\end{Verbatim}
\begin{description}
\item[\code{map}]
A map expression, not a function. The map expression can expect a list of keys in \code{map.keys} and list of values in \code{map.values}.

\item[\code{reduce}]
Can be null if only a map job. If not,reduce should be an expression with three attributes
\begin{description}
\item[\code{pre}]
Called for a new key, but no values have been read. The key is present in \code{reduce.key}.

\item[\code{reduce}]
Called for reducing the incoming values. The values are in a list called \code{reduce.values}

\item[\code{post}]
Called when all the values have been sent.

\end{description}

\item[\code{combiner}]
Uses a combiner if TRUE. If so, then \code{reduce.values} present in the \code{reduce\$reduce} expression will be a \emph{subset} of values.

\item[\code{setup}]
An expression that can be called to setup the environment. Called once for every task.
It can be a list of two attributes \code{map} and \code{reduce} which are expressions to be run in the map and reduce stage. If a single expression then that is run for both map and reduce

\item[\code{cleanup}]
Same as for \code{setup}, run when all work for a task is complete.

\item[\code{ifolder}]
A folder or file to be processed. Can be a vector of strings.

\item[\code{ofolder}]
The folder to store output in. Side effects will be copied here.

\item[\code{inout{}`}]\begin{description}
\item[A vector of input type and output type.]\begin{description}
\item[\code{text}]
indicates Text Format. Use \code{mapred.field.separator} to seperate the elements of a vector.

\end{description}

\item[\code{sequence}]
is a sequence format. Outputs in this form /can/ be used as an input.

\item[\code{binary}]
is a simple binary format consisting of key-length, key data, value-length, value data where the lengths are integers in network order. Though \emph{much} faster than sequence in terms of reading in data, it \emph{cannot} be used an input to a map reduce operation.

\end{description}

\item[\code{shared}]
A vector of files on the HDFS that will be copied to the working directory of the R program. These files can then be loaded as easily as \code{load(filename)} (removed leading path)

\item[\code{jarfiles}]
Copy jar files if required. Experimental, probably doesn't work.

\item[\code{copyFiles}]
For side effects to be copied back to the DFS, set this to TRUE, otherwise they wont be copied.

\item[\code{mapred}]
Set Hadoop options here and RHIPE options.

\item[\code{jobname}]
the jobname, if not given, then current date and time is the job title.

\end{description}


\section{RHIPE Options}
\begin{description}
\item[\textbf{rhipe\_stream\_buffer}]
The size of the STDIN buffer used to write data to the R process(in bytes)
\emph{default:} 10*1024 bytes

\item[\textbf{mapred.textoutputformat.separator}]
The text that seperates the key from value when \code{inout{[}2{]}} equals text.
\emph{default:} Tab

\item[\textbf{mapred.field.separator}]
The text that seperates fields when \code{inout{[}2{]}} equals text.
\emph{default:} Space

\item[\textbf{rhipe\_reduce\_buff\_size}]
The maximum length of \code{reduce.values}
\emph{default:} 10,000

\item[\textbf{rhipe\_map\_buff\_size}]
The maximum length of \code{map.values} (and \code{map.keys})
\emph{default:} 10,000

\end{description}


\section{Status, Counters and Writing Output}


\subsection{Status}

To update the status use \code{rhstatus} which takes a single string e.g \code{rhstatus("Nice")}
This will also indicate progress.


\subsection{Counter}

To update the counter C in the group G with a number N, user \code{rhcounter(G,C,N)}
where C and G are strings and N is a number.


\subsection{Output}

To output data use \code{rhcollect(KEY,VALUE)} where KEY and VALUE are R objects that can be serialized by \code{rhsz} (see the misc page). If one needs to send across complex R objects e.g the KEY is a function, do something like \code{rhcollect(serialize(KEY,NULL),VALUE)}


\section{Side Effect files}

Files written to \code{tmp/} (no leading slash !) e.g \code{pdf("tmp/x.pdf")} will be copied to the output folder.

\resetcurrentobjects
\hypertarget{--doc-rhmisc}{}

\chapter{Miscellaneous Commands}


\section{Introduction}

This is a list of supporting functions for reading, writing sequence files and
manipulating files on the Hadoop Distributed File System (HDFS).


\section{Serialization}


\subsection{rhsz}

\begin{Verbatim}[commandchars=@\[\]]
rhsz @PYGbe[@textless[]-] @PYGap[function](object)
\end{Verbatim}

Serializes a given R object. Currently the only objects that can be serialized
are vectors of Raws,Numerics, Integers, Strings(including NA), Logical(including NA)
and lists of these and lists of lists of these. Attributes are copied to(e.g
names attributes). It appears objects like matrices, factors also get serialized
and unserialized sucessfully.


\subsection{rhuz}

\begin{Verbatim}[commandchars=@\[\]]
rhuz @PYGbe[@textless[]-] @PYGap[function](object)
\end{Verbatim}

Unserializes a raw object returned from \code{rhsz}


\section{HDFS Related}


\subsection{rhsave}

\begin{Verbatim}[commandchars=@\[\]]
rhsave @PYGbe[@textless[]-] @PYGap[function](..., file)
\end{Verbatim}

Saves the objects in \code{...} to \code{file} on the HDFS. All other options are
passed onto the R function \code{save}


\subsection{rhsave.image}

\begin{Verbatim}[commandchars=@\[\]]
rhsave.image @PYGbe[@textless[]-] @PYGap[function](..., file)
\end{Verbatim}

Same as R's \code{save.image}, except that the file goes to the HDFS.


\subsection{rhput}

\begin{Verbatim}[commandchars=@\[\]]
rhput @PYGbe[@textless[]-] @PYGap[function](src,dest,deleteDest@PYGbe[=]@PYGaD[TRUE])
\end{Verbatim}

Copies the file in \code{src} to the \code{dest} on the HDFS, deleting destination if
\code{deleteDest} is TRUE.


\subsection{rhget}

\begin{Verbatim}[commandchars=@\[\]]
rhget @PYGbe[@textless[]-] @PYGap[function](src,dest)
\end{Verbatim}

Copies \code{src{}`{}`(on the HDFS) to {}`{}`dest} on the local. If \code{src} is a directory and \code{dest} exists,
\code{src} is copied inside \code{dest{}`{}`(i.e a folder inside {}`{}`dest}).If not(i.e
\code{dest} does not exist), \code{src}`s contents is copied to a new folder called
\code{dest}.  If \code{src} is a file, and \code{dest} is a directory \code{src} is copied
inside \code{dest} . If \code{dest} does not exist, it is copied to that file

Wildcards allowed

OVERWRITES!


\subsection{rhls}

\begin{Verbatim}[commandchars=@\[\]]
rhls @PYGbe[@textless[]-] @PYGap[function](dir)
\end{Verbatim}

Lists the path at \code{dir}. Wildcards allowed.


\subsection{rhdel}

\begin{Verbatim}[commandchars=@\[\]]
rhdel @PYGbe[@textless[]-] @PYGap[function](dir)
\end{Verbatim}

Deletes file(s) at/in \code{dir}. Wildcards allowed.


\subsection{rhwrite}

\begin{Verbatim}[commandchars=@\[\]]
rhwrite @PYGbe[@textless[]-] @PYGap[function](lo,f,n@PYGbe[=]@PYGaD[NULL],...)
\end{Verbatim}

Writes the list \code{lo}  to the file \code{f}. \code{n} is the number of sequence files
to split the list into.  The default value of \code{n} is
\code{mapred.map.tasks} * \code{mapred.tasktracker.map.tasks.maximum} .


\subsection{rhread}

\begin{Verbatim}[commandchars=@\[\]]
rhread @PYGbe[@textless[]-] @PYGap[function](files,max@PYGbe[=]@PYGaD[NA],batch@PYGbe[=]@PYGaS[100],length@PYGbe[=]@PYGaS[1000])
\end{Verbatim}

Reads files(s) from \code{files} (which could be a directory). Wildcards allowed.

\code{max} is the maximum number of key-values to be read.

\code{batch} is how many to key-value pairs to request from Java in one go.

\code{length} is the initial size of the return list( a larger value will makes
things faster if one is expecting to read in many items ).

The latter two are important when it comes to reading sequence files with many values(100K+), set
batch to a large number and \code{length} to an equally large number to reduce the
number of JNI calls and vector resizes.


\subsection{rhreadBin}

\begin{Verbatim}[commandchars=@\[\]]
rhreadBin @PYGbe[@textless[]-] @PYGap[function](filename, max@PYGbe[=]as.integer(@PYGaS[-1]), bf@PYGbe[=]as.integer(@PYGaS[0]))
\end{Verbatim}

Reads data outputed in `binary' form. \code{max} is the maximum number to read, -1
is all. \code{bf} is the read buffer, 0 implies the os specified default \code{BUFSIZ}

\resetcurrentobjects
\hypertarget{--doc-ec2}{}

\chapter{Using RHIPE on EC2}


\section{Introduction}

There is one 32 bit EC2 AMI with R-2.8, Hadoop 0.21 and the latest RHIPE. \href{http://s3sync.net/wiki}{s3sync} is also present.

The following describes the usage of the EC2 scripts.


\section{Usage}
\begin{itemize}
\item {} 
Get an Amazon EC2 account and confirm the ability to start and instance from the command line (using ec2-tools).

\item {} 
Unzip the rhipe-ec2 distribution (see the downloads page)

\item {} 
OPTIONS

\end{itemize}

In \code{bin/hadoop-ec2-env.sh} template there are several options:
\begin{description}
\item[AWS\_ACCOUNT\_ID]
fill this from the Amazon Account Identifiers

\item[AWS\_ACCESS\_KEY\_ID]
same as above

\item[AWS\_SECRET\_ACCESS\_KEY]
same as above

\item[R\_USER\_FILE]
a URL to an R script. This file is executed on machine boot up. Useful to install R packages. Read \code{bin/hadoop-ec2-env.sh.template} for details.

\item[INSTANCE\_TYPE]
choose the Amazon machine instance type. For details, go to
\href{http://aws.amazon.com/ec2/instance-types/}{http://aws.amazon.com/ec2/instance-types/}

\end{description}
\begin{itemize}
\item {} 
Save the file as \code{bin/hadoop-ec2-env.sh}

\end{itemize}


\section{Some launch commands}
\begin{itemize}
\item {} 
launch

\end{itemize}

\begin{Verbatim}[commandchars=@\[\]]
bin/hadoop-ec2 launch-cluster clustername number-of-workers
\end{Verbatim}

Replace clustername with the name of the cluster and number-of-workers with the number of workers. Use Elasticfox to check all the instances are running, this can some time.
\begin{itemize}
\item {} 
login

\end{itemize}

\begin{Verbatim}[commandchars=@\[\]]
bin/hadoop-ec2 login clustername
\end{Verbatim}
\begin{itemize}
\item {} 
terminate

\end{itemize}

\begin{Verbatim}[commandchars=@\[\]]
bin/hadoop-ec2 terminate-cluster clustername
\end{Verbatim}
\begin{itemize}
\item {} 
You can check the status of jobs at masterip:50030 in your web browser.

\end{itemize}


\section{Useful tools}
\begin{description}
\item[\href{http://www.s3fox.net/}{s3fox}]
A S3 file browser that works within Firefox.

\item[\href{http://sourceforge.net/projects/elasticfox/}{Elasticfox}]
EC2 management tools, a Firefox add-on.

\end{description}

\resetcurrentobjects
\hypertarget{--doc-examples}{}

\chapter{Examples}


\section{\texttt{rhlapply}}


\subsection{Simple Example}

Take a sample of 100 iid observations Xi from N(0,1). Compute the mean of the eight closest neighbours to X1. This is repeated 1,000,000 times.

\begin{Verbatim}[commandchars=@\[\],numbers=left,firstnumber=1,stepnumber=1]
nbrmean @PYGbe[@textless[]-] @PYGap[function](r){
  d @PYGbe[@textless[]-] matrix(rnorm(@PYGaS[200]),ncol@PYGbe[=]@PYGaS[2])
  orig @PYGbe[@textless[]-] d@PYGZlb[]@PYGaS[1],@PYGZrb[]
  ds @PYGbe[@textless[]-] sort(apply(d,@PYGaS[1],@PYGap[function](r) sqrt(sum((r@PYGbe[-]orig)@PYGbe[@textasciicircum[]]@PYGaS[2])))@PYGZlb[]@PYGaS[-1]@PYGZrb[])@PYGZlb[]@PYGaS[1]:@PYGaS[8]@PYGZrb[]
  mean(ds)
}
trials @PYGbe[@textless[]-] @PYGaS[1000000]
\end{Verbatim}

\textbf{One Machine}

\code{trials} is 1,000,000

\begin{Verbatim}[commandchars=@\[\]]
system.time({r @PYGbe[@textless[]-] sapply(@PYGaS[1]:trials, nbrmean)})
 user   system  elapsed
 @PYGaS[1603.414]    @PYGaS[0.127] @PYGaS[1603.789]
\end{Verbatim}

\textbf{Distributed, output to file}

\begin{Verbatim}[commandchars=@\[\]]
mapred @PYGbe[@textless[]-] list(mapred.map.tasks@PYGbe[=]@PYGaS[1000])
r @PYGbe[@textless[]-] rhlapply(@PYGaS[1000000], fun@PYGbe[=]nbrmean,ofolder@PYGbe[=]@PYGaB["]@PYGaB[/test/one"],mapred@PYGbe[=]mapred)
rhex(r)
\end{Verbatim}

Which took 7 minutes on a 4 core machine running 6 JVMs at once.


\subsection{Using Shared Files and Side Effects}

\begin{Verbatim}[commandchars=@\[\],numbers=left,firstnumber=1,stepnumber=1]
h@PYGbe[=]rhlapply(length(simlist)
  ,func@PYGbe[=]@PYGap[function](r){
    @PYGaf[@#@# do something from data loaded from session.Rdata]
    pdf(@PYGaB["]@PYGaB[tmp/a.pdf"])
    plot(animage)
    dev.off()},
  setup@PYGbe[=]expression({
    load(@PYGaB["]@PYGaB[session.Rdata"])
  }),
  hadoop@PYGbe[=]list(mapred.map.tasks@PYGbe[=]@PYGaS[1000]),
  shared.files@PYGbe[=](@PYGaB["]@PYGaB[/tmp/session.Rdata"]))
\end{Verbatim}

Here \code{session.Rdata} is copied from HDFS to local temporary directories (making for faster reads). This
is a useful idiom for loading code that the \code{rhlapply} function might depend on. For example, assuming the image is not \emph{huge}

\begin{Verbatim}[commandchars=@\[\],numbers=left,firstnumber=1,stepnumber=1]
rhsave.image(@PYGaB["]@PYGaB[/tmp/myimage.Rdata"])
rhlapply(N,@PYGap[function](r) {
  object @PYGbe[@textless[]-] dataset@PYGZlb[]@PYGZlb[]r@PYGZrb[]@PYGZrb[]
  G(object)
},setup@PYGbe[=]expression({load(@PYGaB["]@PYGaB[myimage.Rdata"])}))
\end{Verbatim}

In the above example, I wish to apply the \code{G} to every element in \code{dataset}.


\section{\texttt{rhmr}}


\subsection{Word Count}

Generate the words, 1 word every line

\begin{Verbatim}[commandchars=@\[\]]
rhlapply(@PYGaS[10000],@PYGap[function](r) paste(sample(letters@PYGZlb[]@PYGaS[1]:@PYGaS[10]@PYGZrb[],@PYGaS[5]),collapse@PYGbe[=]@PYGaB["]@PYGaB["]),output.folder@PYGbe[=]'@PYGbe[/]tmp@PYGbe[/]words')
\end{Verbatim}

Word count using the sequence file

Run it

\begin{Verbatim}[commandchars=@\[\]]
z @PYGbe[@textless[]-] rhmr(map@PYGbe[=]m,reduce@PYGbe[=]r,inout@PYGbe[=]c(@PYGaB["]@PYGaB[sequence"],@PYGaB["]@PYGaB[sequence"]),
       ifolder@PYGbe[=]@PYGaB["]@PYGaB[/tmp/words"],ofolder@PYGbe[=]'@PYGbe[/]tmp@PYGbe[/]wordcount')
 rhex(z)
\end{Verbatim}


\subsection{Subset a file}

We can use this RHIPE to subset files. Setting \code{mapred.reduce.tasks} to 5 writes the subsetted data across 5 files (even though we haven't provided a reduce task)

\begin{Verbatim}[commandchars=@\[\],numbers=left,firstnumber=1,stepnumber=1]
m @PYGbe[@textless[]-] expression({
  @PYGap[for](x in map.values){
    y @PYGbe[@textless[]-] strsplit(x,@PYGaB["]@PYGaB[ +"])@PYGZlb[]@PYGZlb[]@PYGaS[1]@PYGZrb[]@PYGZrb[]
    @PYGap[for](w in y) rhcollect(w,T)
  }})
z @PYGbe[@textless[]-] rhmr(map@PYGbe[=]m,inout@PYGbe[=]c(@PYGaB["]@PYGaB[text"],@PYGaB["]@PYGaB[binary"]),
    ifolder@PYGbe[=]@PYGaB["]@PYGaB[X"],ofolder@PYGbe[=]'Y',mapred@PYGbe[=]list(mapred.reduce.tasks@PYGbe[=]@PYGaS[5]))
rhex(z)
\end{Verbatim}

\resetcurrentobjects
\hypertarget{--doc-FAQ}{}

\chapter{FAQ}
\begin{enumerate}
\item {} 
Local Testing?

\end{enumerate}

Easily enough. In \code{rhmr} or \code{rhlapply}, set \code{mapred.job.tracker} to
`local' in the \code{mapred} option of the respective command. This will
use the local jobtracker to run your commands.

However keep in mind,
\code{shared.files} will not work, i.e those files will not be copied to the
working directory and side effect files will not be copied back.
\begin{enumerate}
\item {} 
Speed?

\end{enumerate}

Similar to Hadoop Streaming. The bottlenecks are writing and reading to STDIN
pipes and R.

\resetcurrentobjects
\hypertarget{--doc-ProtoBuffers}{}

\chapter{Protobuffer and R}

A package called rprotobuf which implements a simple serialization using Googles
protocol buffers{[}1{]}.  The package also includes some miscellaneous functions for
writing/reading variable length encoded integers, and Base64 encoding/decoding
related functions.  The package can be downloaded from
\href{http://ml.stat.purdue.edu/rprotobuf\_1.0.tar.gz}{http://ml.stat.purdue.edu/rprotobuf\_1.0.tar.gz} it requires one to install libproto
(Googles protobuffer library)

\emph{Brief Description}

The R objects that can be serialized are numerics,integers,strings, logicals,
raw,nulls and lists.  Attributes of the aforementioned are preserved. NA is also
preserved(for the above) As such, the objects include factors and matrices.  The proto file can be
found in the source.

\emph{Todo}:

Very possible though needs some reading on my part: it is possible to
extend this to serialize functions, expressions, environments and several
other objects.  However that is some time in the future.

\emph{Download}
\href{http://ml.stat.purdue.edu/rprotobuf\_1.0.tar.gz}{http://ml.stat.purdue.edu/rprotobuf\_1.0.tar.gz}

{[}1{]} \href{http://code.google.com/apis/protocolbuffers/docs/overview.html}{http://code.google.com/apis/protocolbuffers/docs/overview.html}


\renewcommand{\indexname}{Module Index}
\printmodindex
\renewcommand{\indexname}{Index}
\printindex
\end{document}
